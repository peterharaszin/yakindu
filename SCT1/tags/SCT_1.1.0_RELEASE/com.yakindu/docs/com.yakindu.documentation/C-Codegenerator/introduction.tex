%%%%%%%%%%%%%%%%%%%%%%%%%%%%%%%%%%%%%%%%%%%%%%%%%%%%%%%%%%%%%%%%%%%%%%%%%%%
% Copyright (c) 2010 committers of YAKINDU and others.
% All rights reserved. This program and the accompanying materials
% are made available under the terms of the Eclipse Public License v1.0
% which accompanies this distribution, and is available at
% http://www.eclipse.org/legal/epl-v10.html
%
% Contributors:
%     committers of YAKINDU - initial API and implementation
%%%%%%%%%%%%%%%%%%%%%%%%%%%%%%%%%%%%%%%%%%%%%%%%%%%%%%%%%%%%%%%%%%%%%%%%%%%
\section{Introduction}

The creation of a state machine is shown in YAKINDU Tutorial in detail. However,
what is left is how the state machine models can be used to create source code.
The purpose of this document is to give an overview how the code is created and
how the created code can be integrated into an existing project.

The C source code generator, shipped with the YAKINDU release, is optimized for
small embedded systems with certain restrictions, like small RAM/ROM, ANSI-C
restrictions and MISRA rules (i.e. no heap usage, no function pointers). These
restrictions are mandatory for many tasks e.g. in the automotive area.

Currently, the \textbf{YAKINDU C source code generator} is under heavy
development as the other YAKINDU features, too. So the interfaces are not fixed
yet and will probably change in the near future.

This document guides through an example scenario on an \textbf{Display3000}
development board. It uses an Actmel ATMega128 CPU, 128 kByte Flash and 4 kByte
RAM.

The Traffic Light example, which is discussed in this section, is included into
the YAKINDU examples and can be installed as described in section
\ref{sec:exampleProjects}.


\clearpage