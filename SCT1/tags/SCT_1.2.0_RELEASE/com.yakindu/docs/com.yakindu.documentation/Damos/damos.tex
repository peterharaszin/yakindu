%%%%%%%%%%%%%%%%%%%%%%%%%%%%%%%%%%%%%%%%%%%%%%%%%%%%%%%%%%%%%%%%%%%%%%%%%%%
% Copyright (c) 2010 committers of YAKINDU and others.
% All rights reserved. This program and the accompanying materials
% are made available under the terms of the Eclipse Public License v1.0
% which accompanies this distribution, and is available at
% http://www.eclipse.org/legal/epl-v10.html
%
% Contributors:
%     committers of YAKINDU - initial API and implementation
%%%%%%%%%%%%%%%%%%%%%%%%%%%%%%%%%%%%%%%%%%%%%%%%%%%%%%%%%%%%%%%%%%%%%%%%%%%
In the field of engineering, technical systems are often described by the concept of a dynamical system, 
which is a mathematical formalization of a time-dependent process (e.g. swinging pendulum). The prevalent 
method for modeling dynamical systems is by using block diagrams, which specify the signal flow between 
the components of the system.

The Dynamical Systems Modeler (DAMOS) supports this modeling approach and consists of three components:
\begin {itemize}
\item a block diagram editor
\item a simulator
\item and code generator for efficient C code 
\end{itemize}

DAMOS itself is very modular and can be extended by custom block types and also the simulator and code generator can be customized. 